\documentclass[14pt]{article}

\usepackage[margin=1in]{geometry}
\usepackage[utf8]{inputenc}
\usepackage{graphicx}
\usepackage{vhistory}

\title{Compiler Conrstuction Report\\Prolog}
\author{Ashjan Bakri(1132269)\\Tareq Kirresh(1142828)\\Rula
Shoubaki(1131530)\\Mohannad Jamal(1140643)\\Yousef Assaf(1090985)}
\begin{document}
\begin{figure}
\centering
  \includegraphics[width=7cm]{LOGO.png}
\end{figure}
\maketitle
\newpage
\tableofcontents 
\newpage 
\section{History}

Prolog was designed in the 1970s by Alain Colmerauer and a team of 
researchers with the idea – new at that time – that it was possible to 
use logic to represent knowledge and to write programs. More precisely,
Prolog uses a subset of predicate logic and draws its structure from 
theoretical works of earlier logicians such as Herbrand (1930) and 
Robinson (1965) on the automation of theorem proving. Prolog was 
originally intended for the writing of natural language processing 
applications. Because of its conciseness and simplicity, it became 
popular well beyond this domain and now has adepts in areas such as:
\begin{itemize}
\item Formal logic and associated forms of programming
\item Reasoning modeling
\item Database programming
\item Planning, and so on.
\end{itemize}
Colmerauer started his work at the University of Montréal, and a first 
version of the language was implemented at the University of Marseilles 
in 1972. Colmerauer and Roussel (1996) tell the story of the birth of 
Prolog, including their try-and-fail experimentation to select tractable
algorithms from the mass of results provided by research in logic.


In 1995, the International Organization for Standardization (ISO) 
published a standard on the Prolog programming language. Standard Prolog
(Deransart et al.1996) is becoming prevalent in the Prolog community and
most of the available implementations now adopt it, either partly or 
fully.
 
 
Prolog evolved out of research at the University of Aix-Marseille back 
in the late 60's and early 70's. Alain Colmerauer and Phillipe Roussel,
both of University of Aix-Marseille, colaborated with Robert Kowalski of
the University of Edinburgh to create the underlying design of Prolog as
we know it today. Kowalski contributed the theoretical framework on 
which Prolog is founded while Colmerauer's research at that time 
provided means to formalize the Prolog language. 


1972 is referred to by most sources as the birthdate of Prolog. Since 
its birth it has branched off in many different dialects. Two of the 
main dialects of Prolog stem from the two Universities of its origin: 
Edinburgh and Aix-Marseille. At this time the first Prolog interpreter 
was built by Roussel. The first Prolog compiler was credited to 
David Warren, an expert on Artificial Intelligence at the University of
Edinburgh. 

To this day Prolog has grown in use throughout North America and Europe.
Prolog was used heavily in the European Esprit program and in Japan 
where it was used in building the ICOT Fifth Generation Computer 
Systems Initiative. The Japanese Government developed this project 
in an attempt to create intelligent computers. Prolog was a main player 
in these historical computing endeavours. 

Prolog became even more pervasive when Borland's Turbo Prolog was 
released in the 1980's. The language has continued to develop and be 
used by many scientists and industry experts. Now there is even and 
ISO Prolog standardization (1995) where all of its individual parts 
have been defined to ensure that the core of the language remains fixed. 
\newpage 
\section{Language Syntax and Semantics}
\subsection{Data Types}
Prolog's single data type is the term. Terms are either atoms, numbers, 
variables or compound terms.
\begin{itemize}
\item An atom is a general-purpose name with no inherent meaning.
Examples of atoms include x, red, 'Taco', and 'some atom'.

\item Numbers can be floats or integers.
ISO standard compatible Prolog systems can check the Prolog flag 
"bounded".Most of the major Prolog systems support arbitrary 
precision integer numbers.

\item Variables are denoted by a string consisting of letters, 
numbers and underscore characters, and beginning with an 
upper-case letter or underscore. Variables closely resemble variables in
logic in that they are placeholders for arbitrary terms.

\item A compound term is composed of an atom called a "functor" and a 
number of "arguments", which are again terms. Compound terms are 
ordinarily written as a functor followed by a comma-separated list of 
argument terms, which is contained in parentheses.
The number of arguments is called the term's arity.
An atom can be regarded as a compound term with arity zero.
Examples of compound terms are truck\_year 
\begin{verbatim}('Mazda', 1986)\end{verbatim} and 
'Person\_Friends' \begin{verbatim}(zelda,[tom,jim]).\end{verbatim}
Special cases of compound terms:
\begin{itemize}
\item Lists: A List is an ordered collection of terms.
It is denoted by square brackets with the terms separated by commas or 
in the case of the empty list, []. For example, [1,2,3] or 
[red,green,blue].

\item Strings: A sequence of characters surrounded by quotes is 
equivalent to a list of (numeric) character codes,
generally in the local character encoding, or Unicode if the system
supports Unicode. For example, "to be, or not to be".
\end{itemize}
\end{itemize}
\subsection{Rules and Facts}
Prolog programs describe relations, defined by means of clauses.
Pure Prolog is restricted to Horn clauses.
There are two types of clauses: facts and rules. A rule is of the form
\begin{verbatim}Head :- Body.\end{verbatim}
and is read as "Head is true if Body is true". A rule's body consists 
of calls to predicates, which are called the rule's goals.
The built-in predicate ,/2 (meaning a 2-arity operator with name ,)
denotes conjunction of goals, and ;/2 denotes disjunction.
Conjunctions and disjunctions can only appear in the body,
not in the head of a rule.

\newpage
\section{Examples}
\subsection{Simple Facts in a Knowledge Base}
Lets say we have this knowledge base:
\begin{verbatim}
	woman(mia).
	woman(jody).
	woman(yolanda).
	playsAirGuitar(jody).
	party.
\end{verbatim}
by posing the query :
\begin{verbatim}
	woman(mia).
\end{verbatim}
we will get
\begin{verbatim}	yes\end{verbatim}
This is because we explicitly recorded in the knowledge base above.
Similarly, if we pose this query :
\begin{verbatim}
	playsAirGuitar(jody).
\end{verbatim} 
we will, again, get
\begin{verbatim}	yes\end{verbatim}
Because once more, this is a fact we stated in the knowledge base we are 
querying. Say we pose this query :
\begin{verbatim}
	tatooed(jody).
\end{verbatim}
We will get the answer
\begin{verbatim}	no\end{verbatim}
This is 
because we do not have this statement in our knowledge base, so Prolog
will return no, since it doesn't know if it is true or not.
\newpage 
\subsection{Simple Rules in a Knowledge Base}
Say we have this Knowledge Base :
\begin{verbatim}
	happy(yolanda).
	listens2Music(mia).
	listens2Music(yolanda):- happy(yolanda).
	playsAirGuitar(mia):- listens2Music(mia).
	playsAirGuitar(yolanda):- listens2Music(yolanda).
\end{verbatim}
The First 2 statements are facts, the last 3 statements are 
rules. Suppose we pose this query :
\begin{verbatim}
	playsAirGuitar(mia).
\end{verbatim}
Prolog will respond yes. Why? Well, although it can’t find 
playsAirGuitar(mia) as a fact explicitly recorded in KB2, 
it can find the rule
\begin{verbatim}
playsAirGuitar(mia):- listens2Music(mia).
\end{verbatim}
Moreover, KB2 also contains the fact listens2Music(mia).
Hence Prolog can  deduce that playsAirGuitar(mia) . 
lets pose this query :
\begin{verbatim}
	playsAirGuitar(yolanda).
\end{verbatim}
Prolog would respond yes. Why? Well, first of all,
by using the fact happy(yolanda) and the rule
\begin{verbatim}
listens2Music(yolanda):- happy(yolanda).
\end{verbatim}
Prolog can deduce the new fact listens2Music(yolanda).
This new fact is not explicitly recorded in the knowledge base — it is 
only implicitly present (it is inferred knowledge). Nonetheless,
Prolog can then use it just like an explicitly recorded fact. In
particular, from this inferred fact and the rule
\begin{verbatim}
playsAirGuitar(yolanda):- listens2Music(yolanda).
\end{verbatim}
it can deduce playsAirGuitar(yolanda),
which is what we asked it. Summing up: any fact produced by an 
application of modus ponens can be used as input to further rules.
By chaining together applications of modus ponens in this way,
Prolog is able to retrieve information that logically follows from the
rules and facts recorded in the knowledge base.
\newpage 
\subsection{Defining a Concept with a Knowledge Base}
Say we have this knowledge base :
\begin{verbatim}
	loves(vincent,mia).
	loves(marsellus,mia).
	loves(pumpkin,honey_bunny).
	loves(honey_bunny,pumpkin).

	jealous(X,Y):- loves(X,Z), loves(Y,Z).
\end{verbatim}
In effect, it is defining a concept of jealousy. It says that an 
individual X will be jealous of an individual Y if there is some 
individual Z that X loves, and Y loves that same individual Z too. The 
key thing to note is that this is a general statement: it is not stated 
in terms of mia , or pumpkin , or anyone in particular — it’s a 
conditional statement about everybody in our little world. Lets pose 
this query :
\begin{verbatim}
	jealous(marsellus,W).
\end{verbatim}
This query asks: can you find an individual W such that Marsellus is 
jealous of W ? Vincent is such an individual.
If you check the definition of jealousy,
you’ll see that Marsellus must be jealous of Vincent,
because they both love the same woman, namely Mia. So Prolog will return
the value :
\begin{verbatim}
	W = vincent
\end{verbatim}
\subsection{Recursive Definitions in a Knowledge Base}
Say we have this definition in a knowledge base :
\begin{verbatim}
	gcd(u,v,u) :- v = 0.
		gcd(u,v,x) :- v > 0,
		y is u mod v,
		gcd(v,y,x).
\end{verbatim}
What this effectively does is express, using first order logic, the 
states we go through in order to find the greatest common divisor 
for a number recursively. 
\newpage
\section{Modern Uses of Prolog}
\subsection{Artificial intelligence}

Artificial intelligence is the branch of computer science concerned with
making computers behave like humans.
Prolog is a programming language centered around a small set of basic 
mechanisms, including pattern matching, tree-based data structuring and 
automatic backtracking. This small set constitutes a surprisingly 
powerful and flexible programming framework. Prolog is especially well 
suited for problems that involve objects - in particular,
structured objects - and relations between them. Features like this make
Prolog a powerful language for artificial intelligence (AI) and 
non-numerical programming in general.

Prolog being easier to use for teaching students Al. Lisp has been the 
primary language of artificial intelligence for many years, but it is a 
low-level language, too low for most students. Prolog has three positive
features that give it key advantages over Lisp. First, Prolog syntax and
semantics are much closer to formal logic, the most common way of 
representing facts and reasoning methods used in the artificial 
intelligence literature. Second, Prolog provides automatic backtracking,
a feature making for considerably easier "search", the most central of 
all artificial intelligence techniques. Third, Prolog supports 
multi-directional (or multi-use) reasoning, in which arguments to a 
procedure can freely be designated inputs and outputs in different 
ways in different procedure calls, so that the same procedure 
definition can be used for many different kinds of reasoning.
Besides this, new implementation techniques have given current 
versions of Prolog close in speed to Lisp implementations,
so efficiency is no longer a reason to prefer Lisp.

\subsection{Automated Theorem Proving}

Automated Theorem Proving, also known as ATP or automated deduction, is 
a subfield of automated reasoning and mathematical logic dealing with 
proving mathematical theorems by computer programs. Automated reasoning 
over mathematical proof was a major impetus for the development of 
computer science.

A Prolog technology theorem prover (PTTP) is an extension of Prolog that
is complete for the full first-order predicate calculus. It differs from
Prolog in its use of unification with the occurs check for soundness,
the model-elimination reduction rule that is added to Prolog inferences
to make the inference system complete, and depth-first 
iterative-deepening search instead of unbounded depth-first search to 
make the search strategy complete. A Prolog technology theorem prover 
has been implemented by an extended Prolog-to-LISP compiler that 
supports these additional features. It is capable of proving theorems 
in the full first-order predicate calculus at a rate of thousands 
of inferences per second.

\subsection{Expert Systems}
In Artificial Intelligence, an expert system is a computer system that 
emulates the decision-making ability of a human expert. Expert systems 
are designed to solve complex problems by reasoning about knowledge, 
represented mainly as if–then rules rather than through conventional 
procedural code. 
Features useful for expert systems:
\begin{itemize}
\item Inputs and updates   
During a query evaluation data can be input from a user, and assertions 
and rules can be added to the program, using special primitive 
relations. The evaluation of read(x) will cause x to be bound to the 
next term typed at the input terminal. The evaluation of assert(x) will 
cause the term which is the current binding of x to be added as a new 
rule. Thus, the rule :
\begin{verbatim}
	Ask-about(c) if print (c, "?") & read (ans)
		      & ans = Yes & assert (c),
\end{verbatim}
used to try to answer the query :
\begin{verbatim}
	Ask-about (Dirt is-fault-with Carburettor),
\end{verbatim}
will print the message :
\begin{verbatim}
	Dirt is-fault-with Carburettor?, 
\end{verbatim}
read the next term, and if it is the constant Yes it will add :
\begin{verbatim}
	Dirt is-fault-with Carburettor
\end{verbatim}
as a new assertion about the is-fault-with relation. Where this 
assertion is added is important. It can be added at the beginning of 
the list of clauses for the relation, or the end, or at some 
intermediate position. In this situation we would like it added at the 
beginning. Where it is added is an option that can be specified by the 
programmer. We shall not go into the details of this. 
\item Dynamic data base 
The rule :
\begin{verbatim}
	u my is-fault-with x if y is-part-of x & u is-fault-with y 
\end{verbatim}
must access assertions giving components and assertions about faults 
with components. We can use the Ask-about rule to allow the assertions 
about faults to be added dynamically as we try to solve the problem of 
finding a fault. 

Instead of assertions about known faults with components we include in 
the initial data base only assertions about possible faults, knowledge 
that expert should have. We then include the rule :
\begin{verbatim}
	 u is-a-fault-with y if u is-a-poss-fault-with y & Ask-about (u is-a-fault-with y) 
\end{verbatim}
Let us pause for a moment to consider the behavior of our fault finder.
When asked to find a fault with some device with a query
\begin{verbatim}
	 u is-fault-with Device
\end{verbatim}
the use of the first rule for faults will cause the fault finder to walk
over the structure of Device as described by the is-part-of assertions.
When it reaches an atomic part it will query the user concerning 
possible faults with this component as listed in the is-poss-fault-with 
assertions. It will continue in this way, backtracking up and down the 
structure, until a fault is reported. As it currently stands, our expert
system helps the user to look for faults. 

\item Generation of lemmas 
Sometimes it is useful to record the successful evaluation of an atomic 
query B by adding its derived substitution instance as a new assertion.
Thus, suppose we have a rule :
\begin{verbatim}
	R(tL.. ,tn) if Al& . .&Ak 
\end{verbatim}
and we want to generate a lemma each time it is successfully used to 
answer a query R(t' 1, , en). We add an extra assert condition at the 
end of the list of preconditions of the rule :
\begin{verbatim}
	R (tl, . . ,tn) if Al& . .&Ak & assert (R(tl, . . , tn))
\end{verbatim}
If this solves the atomic query with answer substitution s then [R , en*
will be added as new assertion. It will be added at the front of the 
sequence of clauses for R.

By adding asserts we can also simulate the use of rules with 
conjunctive consequences. Suppose that we know that both B and B' are 
implied by the conjunction Al\& . . \&An. Normally we would just 
include the two rules : 
\begin{verbatim}
	B if Al& . .&An
	B' if Al& . . &An 
\end{verbatim}
in the program. The drawback is that we need to solve Al\&..\&An twice 
in derivations where both B and B' are needed. We can avoid the
duplication by writing the two rules as :
\begin{verbatim}
	B if Al& ..&An&assert(B') 
	B' if Al& . . &An &assert (B) 
\end{verbatim}
The successful use of either rule will add the other consequent as a 
lemma. 

By developing a suitable front end to Prolog we can shield the 
programmer from the details of the lemma generation. We could allow him
to write rules with conjunctive consequents and to specify which rules 
were lemma generation rules. The front end program would expand rules 
with conjunctive consequents into several rules and add the appropriate
asserts to the end of each of these rules. It would also add an assert
to the end of each of the lemma rules.

\item All solutions 

Sometimes we want to find all the answers to a query, not just one 
answer. This is an option in some of the Prologs. Where it is not we can
make use of a meta-rule such as 
\begin{verbatim}
	All (query, term) if query & print (term) & fail.
\end{verbatim}
This will print out [term]s for each answer substitution s to query. 
The "fail" is a condition for which we assume there are no clauses. 
With a slightly modified rule, we can define the relation :
\begin{verbatim}	
	  is-all term such-that query 
\end{verbatim}
that holds when I is the list of all the instantiations of "term" for 
answer substitutions to "query". In DEC-10 Prolog such a relation is now
a primitive of the language. Using it we can write rules such as 
\begin{verbatim}
	I is-a-list-of-faults-with x if 1 is-all u such-that 
		u is-fault-with x 
\end{verbatim}
The use of this rule will result in 1 being bound to a list of all the 
faults with x. We can now consider a very simple extension to our fault 
finder. Instead of asking for a single fault we can ask for a list of 
all the reported faults with corrective actions. We do this with a 
query of the form : 
\begin{verbatim}
	is-all [u a] such-that 
		u is-fault-with D &a is-action-for u. 
\end{verbatim}

To handle this query we must also include in our database a set of 
assertions giving actions for faults, information supplied by the 
expert. An evaluation of this new query will guide the user through the
structure of device D, asking about faults in components. Each reported
fault will be paired with its corrective action. Finally the list of 
pairs [reported-fault corrective-action] will be printed.

\item Term rewriting

A term rewriting system (TRS) is a rewriting system where the 
objects are terms, or expressions with nested sub-expressions. 
Term rewriting is a surprisingly simple computational paradigm
that is based on the repeated application of simplification rules.
It is particularly suited for tasks like symbolic computation,
program analysis and program transformation. Understanding term 
rewriting will help you to solve such tasks in a very effective manner.

\item Type inference: 
Type inference refers to the automatic deduction of the data type of an
expression in a programming language.

\subsection{Automated Planning and scheduling}

Automated planning and scheduling, sometimes denoted as simply AI 
Planning, is a branch of artificial intelligence that concerns the 
realization of strategies or action sequences, typically for execution
by intelligent agents, autonomous robots and unmanned vehicles. Unlike
classical control and classification problems, the solutions are complex
and must be discovered and optimized in multidimensional space. Planning
is also related to decision theory.

\subsection{Natural language processing}

Natural language processing (NLP) is a field of computer science,
artificial intelligence and computational linguistics concerned with the
interactions between computers and human (natural) languages, and, in
particular, concerned with programming computers to fruitfully process
large natural language corpora.

\end{itemize}
\newpage 
\section{Conclusion}
Researchers created Prolog to serve as a way to introduce the use of
predicate logic in programming while it was originally meant to be as
a way for writing programs considering natural language processing.
This made Prolog a programming language with a rich structure in
language and human reasoning. This differentiated it from common
languages such as C.

Prolog depends on a single data type, which can have four forms,
called ‘term’. A term can be either an atom (general-purpose name), a
number (float or integer), a variable (a string that is a placeholder
for arbitrary terms) or a compound term. The compound term is composed
of an atom called a ‘functor’ followed by a list of arguments
separated by commas. Prolog programs describe logic relations using two
types of clauses: facts and rules. A rule’s body consists of calls to
predicates, a rule’s ‘goal’. A fact is a declaration of a true state
of affairs, and any fact is generally treated as true. A Prolog
program is written as a set of facts and sometimes a set of rules, as
well.

This gave Prolog consistency and simplicity to find its way into many
fields beyond its main domain, such as Artificial Intelligence,
Automated Theorem Proving, Automated Planning and Scheduling, and
Natural Language Processing.

\newpage
\section{References} 
\begin{verbatim}
	
http://fileadmin.cs.lth.se/cs/Personal/Pierre_Nugues/ilppp/chapters/appA.pdf.
http://www.mta.ca/~rrosebru/oldcourse/371199/Prolog/history.html. 
http://www.doc.ic.ac.uk/~rak/papers/the%20early%20years.pdf.
https://pdfs.semanticscholar.org/1dba/24320c41fa5eae94c51fd971085bc8176dd7.pdf
https://link.springer.com/article/10.1007%2FBF00297245?LI=true
https://en.wikipedia.org/wiki/Automated_theorem_proving#First_implementations
https://calhoun.nps.edu/bitstream/handle/10945/36984/Rowe_book_AI_thr_Prolog_preface.pdf https://softwareengineering.stackexchange.com/questions/84407/why-is-Prolog-good-for-ai-programming
https://en.wikipedia.org/wiki/Expert_system
https://en.wikipedia.org/wiki/Natural_language_processing 
https://en.wikipedia.org/wiki/Automated_planning_and_scheduling 
https://en.wikipedia.org/wiki/Type_inference https://pdfs.semanticscholar.org/1dba/24320c41fa5eae94c51fd971085bc8176dd7.pdfhttps://en.wikipedia.org/wiki/Prolog
lpn.swi-Prolog.org/lpnpage.php?pagetype=html&pageid=lpn-htmlse1
https://github.com/TareqK/Compiler-Construction-Course/blob/master/Notes/introduction.markdown
\end{verbatim}
\end{document}
